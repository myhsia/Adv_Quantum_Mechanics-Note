%  !TeX root = ../main.tex

\chapter{Symmetry}

\paragraph{What is symmetry?}

\begin{enumext}
  \item Leibniz: Symmetry is \emph{indiscernability of
    \sout{differences} changes \textrightarrow transformations}.
  \item Physical: invariance (under) transformations.
\end{enumext}

\subparagraph{Action part.}

V is the Hilbert space, and we have a map $V \mapsto V$, which is
the operator. The map is invertible.
Transformation stands for something like $X \mapsto \Omega(X)$,
i.e., map something to something else, but ``something else''
is in the same space, such as mapping a vector to a vector.
For example,
\[
\begin{cases*}
  \ket|\psi>  \mapsto \hat\Omega \ket|\psi>,& Transformation of state\\
  \hat O      \mapsto \hat\Omega \hat O \hat\Omega^{-1},
& Transformation of Operator\\
  \phi(x)     \mapsto (\Omega \phi)(x),     & Path Integral
\end{cases*}
\]

\subparagraph{Invariance part.}

When $X \sim Y$, i.e., $X$ is equivalent to $Y$, then it calles the
invariance.
After the map, we get something equivalent to $X$: $\Omega(x) \sim X$.
The invariance in the context can be, for example
\[
  \text{equations / constraints}\ \Omega(x) \sim X \longrightarrow
  \begin{cases*}
    \Omega \ket|\psi> = \ket|\psi> \upe^{\iu\varphi} &
    Transformation of state\\
    \hat\Omega \hat H \hat\Omega^{-1} = \hat H       &
    Transformation of Operator\\
    \mathcal S[\phi] = \mathcal S[\Omega\phi]        &
    Path Integral
  \end{cases*}
\]
To symmetry, $\Omega(x) \sim X$ is just equations/constraints:
Each equation is a kind of constraint on the object.

\vskip1ex \hrule
\subparagraph{Symmary}

For symmetry constraints, what constraints do is a limited possibility:
it is simplicity, or what understandability comes from.
How this happens is related to \emph{Symmetry \& Group Theory.}

The group is a set: we consider a set of transformation $\Omega_i$
\[
  G = \{\Omega_i\}, \qq{and the operation} \Omega_1 \circ \Omega_2
\]
The operation says that $\circ:\ G \times G \to G$.
Concerning the basic properties of Group
\begin{enumext}
  \item Closure: If $\Omega_{1,2}(X) \sim X$, then the combination
  $\Omega_1(\Omega_2(X)) \sim \Omega_i(X) \sim X$ is also in this set.
  \item Identity: Also okay, to do nothing.
  \item Inverse: We can have $\Omega(X) \sim X$ then do the inverse
  $\Omega^{-1}$ on both side
  \[
    X \sim \Omega^{-1}(X)
  \]
  By equivalent, this is so-called the inflective.
  \item Associativity: Such as by Hamiltonian, etc.
\end{enumext}
Referring to the Group theory, we shall talk about the

\section{Group Representation Theory}

\begin{enumext}
  \item Group: Introduce the space; elementary ``particles''
  \item Representation: We can ``lable'' (name) all the representations
  \begin{enumext}
    \item Different labels, which is closely related to conserved quantities
    (quantum numbers);
    \item Dimension: closely related to degeneracy.
  \end{enumext}
\end{enumext}

\subsection{Translation}

A trivial example is just to move the entire function $\phi(x)$ towards one
direction by $a$, we get
\[
  \phi(x) \mapsto \phi(x - a) \equiv \tilde \phi(x)
\]
where $x \in \mathbb R$. And we can define the translation operator $\hat T_a$
\[
  \phi(x - a) = \tilde \phi(x) = (\hat T_a\phi)(x)
\]
where $\phi(x)$ is the wavefunction, and we can get the new state
\[
  \phi(x) = \braket<x|\phi> \mapsto \phi(x - a) = \braket<x|\hat T_a\phi>
\]
Try to write it in the way of expansion
\[
  \hat T_a\ket|\phi> = \int \d x \ket|x> \braket<x|\hat T_a\phi>
= \int \d x \ket|x> \phi(x - a)
\xlongequal{\tilde x = x - a} \int \d x \ket|x - a> \phi(x)
\]
So, this transformation operator is a linear operator.
Insert the identity
\[
  \ket|\phi> = \int \d x \ket|x> \braket<x|\phi>, \qq{then,}
  \hat T_a\ket|\phi> = \int \d x \hat T_a \ket|x> \phi(x)
\]
In short, we have
\[
  \hat T_a\ket|x> = \ket|x + a>
\]
If $\ket|x>$ refers to a delta function $\delta(x)$, then the transformation
just move it to $\delta(x + a)$.
To write $T_a$ in basis
\[
  \hat T_a = \int \d x \ketbra|x + a><x|
\]
\paragraph{Quiz.}
Try to compute $\hat T_a \hat T_b$.
\[
  \hat T_a \hat T_b = \hat T_{a+b=b_a} = \hat T_b \hat T_a
\]

\subsection{Reflection}

\[
  \phi(x) \mapsto \phi(2a - x)
\]
We call the operation $\rho_a$.
The same logic.
\[
  \hat\rho_a\ket|\phi> = \int \d x \ket|x> \braket<x|\hat\rho_a\phi>
= \int \d x \ket|x> \phi(2a - x) = \int \d x \ket|2a - x> \phi(x)
\]
\textbf{Remember no minus sign here: the integration range also reversed.}
So, we obtain
\[
  \hat \rho_a\ket|x> = \ket|2a - x>
\]
\paragraph{Quiz}
Compute $\hat \rho_a \hat \rho_b$.
\[
  \hat\rho_a \hat\rho_b = \hat T_{2a - ab}
\]

\paragraph{Quiz}
Compute $\hat \rho_a \hat \rho_b \hat \rho_c$.
\[
  \hat\rho_a \hat\rho_b \hat\rho_c = \hat \rho_{a-b+c}
\]
There are some identities
\begin{align}
  \hat\rho_a^2 & = \identity,\\
  \hat\rho_a \hat\rho_b & \neq \hat\rho_b \hat\rho_a, \qq{where $a \neq b$}
\end{align}

\paragraph{Quiz}
Prove if $\hat T_a \hat \rho_b \overset?= \hat\rho_b\hat T_a$.
\[
  \hat T_a = \rho_{a/2} \rho_0
\]
So, $\hat T_a \hat \rho_b = \rho_{a/2+b}$, and
$\hat\rho_b\hat T_a = \rho_{b-a/2}$. They are not equal.

\subsection{Invariance}

\paragraph{With translation}

If we compose
\[
  \hat T_a \ket|\phi> = \upe^{\iu\phi} \ket|\phi>
\]
then, the wavefunction $\braket<x|\phi>$ must be periodic.
\emph{The formula above becomes the eigen equation.}
We can consider it as a complex plane wave $\upe^{\iu kx}$, i.e.,
a series loop, perpendicular to $x$.
Then, the wavelength is proportional to the period.

\paragraph{With Reflection}

Consider the eigen value equation
\[
  \hat\rho_a \ket|\psi> = \upe^{\iu\varphi} \ket|\psi>
\]
and we have the identity $\hat\rho_a^2 = \identity$,
then, the eigenvalue $\lambda^2 = 1$, $\lambda = \pm 1$.

Let $a \in \mathbb R$, which is continuous,
then we can have $\hat T_a \xlongrightarrow{a\to0} \identity$.

Consider a seires of atoms and the mirror planes,
\begin{center}
  \begin{tikzpicture}
    \draw (0,0) -- (7,0);
    \foreach \a in {1,2,...,6}
      {
        \draw (\a,0) circle (.1);
        \draw [dashed] ({\a + .5},1) --++ (0,-2);
      }
    \draw [->] (3.5,-1) --++ (1,0) node [below, midway] {$\hat T_a$};
  \end{tikzpicture}
\end{center}
the two mirror planes is related to $\hat T_a$.
It is not possible to have a $a$ to make $\hat \rho_a \to \identity$.

With the two examples above, we can generalize
\paragraph{General Transformation $\hat\Omega$}

The operator $\hat \Omega$ is unitary, and sometimes it can be anti-unitary,
and it is time-reversal.
The operator acts on states (vectors in Hilbert Space)
\[
  \ket|\psi> \mapsto \hat\Omega\ket|\psi>
= \sum_\alpha \hat\Omega \ket|\alpha> \psi(\alpha)
\]
always induce an action $\hat \Omega$ on operators
\[
  \hat O = \sum_{\alpha\beta} O_{\alpha\beta} \ketbra|\alpha><\beta|
\]
Then, the action $\hat \Omega$ on operators always induce
\[
  \hat O \mapsto
  \sum_{\alpha\beta} O_{\alpha\beta} \ketbra|\Omega\alpha><\Omega\beta|
\]
It is trivial that $\ket|\Omega\alpha> = \Omega\ket|\alpha>$,
but $\bra<\Omega\beta| = \bra<\beta|\Omega^{-1}$.
$\Omega$ is unitary, means that whatever $v$,
\[
  \braket<\Omega u|v> = \braket<\Omega^{-1}\Omega u|\Omega^{-1}v>
= \braket<u|\Omega^{-1}|v>.
\]
So, we have
\[
  \sum_{\alpha\beta} O_{\alpha\beta} \ketbra|\Omega\alpha><\Omega\beta|
= \hat\Omega \hat O\hat\Omega^{-1}
\]

\section{Continuous symmetry and conservation laws}

\subsection{Continuous symmetry}

\underline{Continuous is connected to $\identity$}
\[\begin{array}{ccc}
  \hat\Omega_\theta &
  \xlongleftrightarrow[\hat\Omega_\epsilon = \identity-\iu\epsilon\hat g]
    {\text{infinitesimal}} & \hat g\\
  \downarrow &  & \downarrow\\
  \text{Unitary} & & \text{Hermition}
\end{array}\]
We can give the
\begin{theorem}[Stone's theorem]
  $\forall u$, $v$, the inner product
  \[
    \braket<u|v> = \braket<\Omega_\epsilon u|\Omega_\epsilon v>
  \]
  \begin{proof}
    \[
      \cancel{\braket<u|v>}
    = \cancel{\braket<u|v>}
    + \iu\epsilon(\braket<\hat gu|v> - \braket<u|\hat gv>)
    \]
    then, we obtain
    \[
      \braket<\hat gu|v> = \braket<u|\hat gv>
    \]
    and the adjoint
    \[
      \braket<\hat Ou|v> = \braket<u|\hat O'v>
    \]
    If this is illegal $\forall u$, $v$, then we have
    \[
      \hat O' = \hat O^\dagger = \hat O
    \]
  \end{proof}
\end{theorem}
So, the generator $\hat g$ in this sense can be expressed as
\[
  \hat g = \iu\frac{\hat\Omega_\epsilon - \identity}{\epsilon}
            \bigg|_{\epsilon\to0}
= \iu \odv{\hat\Omega_\theta}\theta \bigg|_{\theta=0}
\]
\paragraph{With translation}

Given the definition
\[
  \hat T_a\ket|x> = \ket|x + a>
\]
then, consider $\hat T_a \hat x \hat T_a^{-1}$.
SInce $\hat x = \int \d x \ketbra|x> x <x|$
\[
  \hat T_a \hat x \hat T_a^{-1}
= \int \d x \ketbra|x - a> x <x + a| = \hat x - a
\]
Assume $a \to \epsilon$, then, we have
\[
  \hat T_\epsilon \hat x \hat T_\epsilon^{-1} = \hat x - \epsilon
\]
Substitute $\hat T_\epsilon = \identity - \iu\epsilon\hat g_T$, we have
\[
  (\identity - \iu\epsilon \hat g_T) \hat x (\identity + \iu\epsilon \hat g_1)
- \iu\epsilon(\hat g_T\hat x - \hat x \hat g_T) = -\epsilon
\]
which means
\[
  [\hat x, \hat g_T] = \iu, \qq{obviously,} \hat g_T = \hat k
\]

\paragraph{With rotation}

Consider the action (in 3D)
\[
  \hat R_{(\bm e_a, \epsilon)} \ket|\bm r> = \ket|?>
\]
Firstly, $\delta\bm r = \epsilon \bm e_a \times \bm r$, then,
\[
  \hat R_{(\bm e_a, \epsilon)} \ket|\bm r>
= \ket|\bm r + \epsilon \hat e_a \times \bm r>
\]
So, we have
\[
  \hat R_{(\bm e_a,\epsilon)} \hat{\bm r} \hat R_{(\bm e_a,\epsilon)}^{-1}
= \hat{\bm r} + \epsilon \bm e_a \times \hat{\bm r}
\]
The generator
$\hat R_{(\bm e_a,\epsilon)} = \identity - \iu\epsilon \hat g_{\bm a}$,
we can simply replace $\epsilon$ with the whole term
\[
  [\hat{\bm r}, \hat g_{\bm e_a}] = \iu\bm e_a \times \hat{\bm r}
\]
The key difference is that the equation is the vector equation, for the scalar,
\[
  [\hat x, \hat g_{\bm e_a}] = \iu(\bm e_a \times \hat{\bm r})_x
\]
To expand it,
\[
  [\hat x, \hat g_{\bm e_a}] = \iu [(e_a)_y\hat z - (e_a)_z \hat y]
\]
Similarly, for $\hat y$ and $\hat z$, the single equation corresponds to 3
sub-equations in total.
Eventually, we will find
\[
  \hat g_{\bm e_a} = \bm e_a \cdot(\hat{\bm r \times \hat{\bm k}})
= \bm e_a \cdot \hat{\bm L}
\]
where $\hat{\bm L} \equiv \hat{\bm r} \times \hat{\bm k}$.
If the rotation is actually the symmetry, then, the projection $\bm e_a$ can be
removed.

If $a$ is no more infinitesimal, i.e.,
to the exponential $U(t) = \upe^{-\iu\hat Ht}$,
we can divide $a$ into $n$-steps, then take the multiplication into power $N$ of
small steps of translations
\[
  \hat T_a = \identity - \iu\epsilon \hat g_T \Rightarrow
  \hat T_a = (\hat T_{a/N}) \xlongequal{N\to\infty}
  \ab(1- \iu\frac aN \hat g_T)^N = \upe^{-\iu a\cancelto{k}{\hat g_T}}
\]
Similarly for the rotation, we have
\[
  \hat R_{(\bm e_a,\epsilon)} = \identity - \iu\epsilon \hat g_{\bm a}
  \Rightarrow
  \hat R_{(\bm e_a,\epsilon)} = \upe^{-\iu\theta \bm e_a \cdot \hat{\bm L}}
= \upe^{-\iu \bm\theta \cdot \hat{\bm L}}
\]
Usually, we say $\bm e_a$ is fixed, then we have only one parameter $\theta$.
Now, we can rewrite this trivially
\[
  \hat R_{\bm\theta} = \upe^{-\iu\bm\theta\hat{\bm L}}
\]
It means we have 3 free parameters, which allows us to do the exponential
mapping.
Functions themselves make the Hilbert space: $\{f(\bm r)\}$
\[
  f(\bm r) = (\hat Rf)(\bm r)
\]

\subsection{Conservation Laws}

\section{Symmetry group representations, degeneracies, inversion and time reversal}

\section{Angular momentum, Lie algebra}

\section{Gauge}
