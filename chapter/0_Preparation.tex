% !TeX root = ../main.tex

\chapter{Preparation: Quantum Mechanics is a physics \emph{theory}}

\paragraph{Part I. Hilbert space (Week 1 -- 4)}

\begin{enumext*}[columns = 3]
  \item(3) Features of HS: basis, linear maps (operators / functionals)
  \item(2) Time evolution: EOM, Dyson series, three pictures
  \item Perturbation theory
\end{enumext*}

\paragraph{Part II. Path integral (Week 5 -- 7)}

\begin{enumext}[columns = 2]
  \item Functional integral, variational principle
  \item Wick rotation, partition function
  \item Correlation functions, Feynman diagrams
\end{enumext}

\paragraph{Part III. Symmerty (Week 8 -- 12)}

\begin{enumext*}[columns = 2]
  \item(2) Continous symmetry and conservations laws
  \item(2) Symmetry group representations, degeneracies, inversion and time reversal
  \item Angular momentum, Lie algebra
  \item Gauge
\end{enumext*}

\paragraph{Part IV. Entanglement (Week 13 -- 16)}

\begin{enumext*}[columns = 15]
  \item(6) Indistinguishable particles, Fock space
  \item(4) Second quantization
  \item(5) Entanglement, Bell's inequality
  \item(15) Open quantum systems, density operator, quantum channel
\end{enumext*}

\section*{\refname}

\begin{refsection}[reference-phx]
  \renewcommand \addcontentsline [3] {}
  \renewcommand \chapter [2] {}
  \nocite{*}
  \printbibliography
\end{refsection}

\section*{Math prerequisite: Math is Very Important!}

To formulate a theory, we need MATH!

To understand a theory, we need PICTURE!

\begin{enumext}[columns = 2]
  \item Calculus (real and complex)
  \item Differential equations
  \item Linear algebra
  \item Abstract algebra (group, field, vector space, etc.)
\end{enumext}

\begin{refsection}[reference-math]
  \renewcommand \addcontentsline [3] {}
  \renewcommand \chapter [2] {}
  \nocite{*}
  \printbibliography
\end{refsection}